% !TEX root = textbook.tex

\chapter{벡터와 행렬}

\section{벡터와 행렬의 정의와 연산} 

\subsection{벡터의 정의와 연산} 

\paragraph{벡터의 정의}

\paragraph{사칙연산} 

\paragraph{선형결합}


\subsection{행렬의 정의와 연산}

\paragraph{행렬의 정의} 

\paragraph{행렬간 사칙연산} 

\paragraph{행렬-벡터간 사칙연산}

\paragraph{Transpose와 Trace}

\paragraph{행렬식}

\subsubsection{행렬의 Elementary Row Operation}

\section{벡터와 행렬의 계산}


\subsection{벡터공간의 정의} 

\subsubsection{선형결합과 선형독립} 

\subsubsection{Span과 기저} 
\paragraph{공간의 차

\subsection{행렬의 Row Space} 

\subsubsection{행렬의 Row Space}
\subsubsection{행렬의 Rank와 Nullity}


\subsubsection{가우스 소거법}
\subsubsection{Row Echelon Form}
\subsubsection{가우스 소거법과 Row Space} 

\subsection{고유값과 고유벡터} 

\subsection{역행렬} 

\subsection{다양한 행렬의 분해} 

\subsubsection{LU Decomposition} 
\subsubsection{QR Decomposition} 
\subsubsection{Singlular Value Decomposition}
\section{행렬과 함수}


\section{벡터와 행렬의 응용} 
\subsection{연립방정식의 해법}
\subsection{Linear Least Square 문제의 해법}
\subsection{기하학적 응용}
\subsection{Support Vector Machine}

\section{일반화된 벡터와 벡터공간}

\subsection{Field의 정의} 

\subsection{벡터와 벡터공간의 정의} 

\subsection{다양한 벡터의 예제}
